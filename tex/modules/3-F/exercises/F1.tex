\begin{problem}{V1}
Let \(V\) be the  set of all real numbers together with the operations \(\oplus\) and \(\odot\) defined by, for any \(x,y\in V\) and \(c\in \IR\),
\begin{align*}
x\oplus y  &= x+y \\
c \odot x &= cx-3(c-1)
\end{align*}
\begin{enumerate}[(a)]
\item Show that \textbf{scalar multiplication} is
      \textbf{associative}: \(a\odot(b\odot x)=(ab)\odot x\) for all scalars \(a,b \in \IR\) and \(x \in V\).
\item Show that scalar multiplication does not distribute over vector addition, i.e. for some scalar \(a \in \IR\) and \(x,y\in V\), \(a \odot (x \oplus y) \neq a \odot x \oplus a \odot y\).
\end{enumerate}
\end{problem}


\begin{problem}{V1}
Let \(V\) be the set of all pairs of real numbers with the operations, for any \((x_1,x_2), (y_1,y_2) \in V\), \(c\in \IR\),
\begin{align*}
(x_1,x_2) \oplus (y_1,y_2) &= (x_1+y_1,x_2+y_2+2x_1y_1) \\
c \odot (x_1,x_2) &= (cx_1, cx_2)
\end{align*}
\begin{enumerate}[(a)]
\item Show that the vector \textbf{addition} \(\oplus\) is \textbf{associative}:
      \((x_1,x_2) \oplus ((y_1,y_2) \oplus (z_1,z_2))=((x_1,x_2)\oplus (y_1,y_2))\oplus (z_1,z_2)\) for all \((x_1,x_2), (y_1,y_2), (z_1,z_2) \in V\).
\item Show that scalar multiplication does not distribute over scalar addition, i.e. for some scalars \(a,b \in \IR\) and \((x_1,x_2) \in V\), \( (a+b) \odot (x_1,x_2) \neq a\odot (x_1,x_2) \oplus b \odot (x_1,x_2) \).
\end{enumerate}
\end{problem}



\begin{problem}{V1}
Let \(V\) be the set of all pairs of real numbers with the operations, for any \((x_1,x_2), (y_1,y_2) \in V\), \(c\in \IR\),
\begin{align*}
(x_1,x_2) \oplus (y_1,y_2) &= (x_1+y_1-1,x_2+y_2-1) \\
c \odot (x_1,x_2) &= (cx_1, cx_2)
\end{align*}
\begin{enumerate}[(a)]
\item Show that this vector space has an \textbf{additive identity} element, i.e. an element
      \(\mathbf{z} \in V\) satisfying \((x,y)\oplus\mathbf{z}=(x,y)\) for every \((x,y) \in V\).
\item Show that scalar multiplication does not distribute over vector addition, i.e. for some \(a \in \IR\) and \( (x_1,x_2), (y_1,y_2) \in V\), \(a \odot \left( (x_1,x_2)\oplus (y_1,y_2) \right) \neq a \odot (x_1,x_2) \oplus a \odot (y_1,y_2) \).
\end{enumerate}
\end{problem}

\begin{problem}{V1}
Let \(V\) be the set of all pairs of real numbers with the operations, for any \((x_1,x_2), (y_1,y_2) \in V\), \(c\in \IR\),
\begin{align*}
(x_1,x_2) \oplus (y_1,y_2) &= (x_1+y_1,x_2+y_2) \\
c \odot (x_1,x_2) &= (0, cx_2)
\end{align*}
\begin{enumerate}[(a)]
\item Show that \textbf{scalar multiplication
      distributes over scalar addition}, i.e. that
      \((c+d)\odot(x_1,x_2)=
      c\odot(x_1,x_2) \oplus d\odot(x_1,x_2)\) for every \(c,d \in \IR\) and \( (x_1,x_2) \in V\).
\item Show that 1 is not a scalar multiplicative identity.
\end{enumerate}
\end{problem}


 \begin{problem}{V1}
 Let \(V\) be the set of all pairs of real numbers with the operations, for any \((x_1,x_2), (y_1,y_2) \in V\), \(c\in \IR\),
 \begin{align*}
 (x_1,x_2) \oplus (y_1,y_2) &= (x_1+y_1,x_2+y_2) \\
 c \odot (x_1,x_2) &= (c^2x_1, c^3x_2)
 \end{align*}
 \begin{enumerate}[(a)]
 \item Show that \textbf{scalar multiplication distributes over
       vector addition}, i.e. that
       \(c\odot((x_1,x_2) \oplus (y_1,y_2))=
       c\odot(x_1,x_2) \oplus c\odot(y_1,y_2)\) for all \(c \in \IR\) and \( (x_1,x_2), (y_1,y_2) \in V\).
 \item Show that scalar multiplication does not distribute over
       scalar addition, i.e. for some \(c,d \in \IR\) and \( (x_1,x_2)\in V\)
       \((c+d)\odot(x_1,x_2)\neq c\odot(x_1,x_2) \oplus d\odot(x_1,x_2)\).
 \end{enumerate}
\end{problem}
%
%
% \begin{problem}{V1}
% Let \(V\) be the set of all polynomials with the operations, for any \(f,g\in V\), \(c\in \IR\),
% \begin{align*}
% f \oplus g &= f^\prime + g^\prime \\
% c \odot f &= c f^\prime
% \end{align*}
% (here \(f^\prime\) denotes the derivative of \(f\)).
% \begin{enumerate}[(a)]
% \item Show that scalar multiplication distributes over
%       vector addition, i.e. for all \(c\in \IR\) and \(f,g \in V\)
%       \(c\odot(f \oplus g)=
%       c\odot f \oplus c\odot g\).
% \item  Show that there is no additive identity element.
% \end{enumerate}
% \end{problem}
% \begin{solution}
% Let \(f,g \in \mathcal{P}\), and let \(c\in \IR\).
% \[c \odot (f \oplus g) = c \odot (f^\prime+g^\prime) =
% c(f^\prime+g^\prime)^\prime = cf^{\prime\ \prime}+cg^{\prime\ \prime} =
% cf^\prime\oplus cg^\prime= c \odot f \oplus c \odot g.\]
% However, this is not a vector space, as there is no zero vector.  Additionally, \(1 \odot f \neq f\) for any nonzero polynomial \(f\).
% \end{solution}
%
%
 \begin{problem}{V1}
 Let \(V\) be the set of all real numbers with the operations, for any \(x,y\in V\), \(c\in \IR\),
 \begin{align*}
 x \oplus y &= \sqrt{x^2+y^2} \\
 c \odot x &= c x
 \end{align*}
 \begin{enumerate}[(a)]
 \item Show that the \textbf{vector addition \(\oplus\) is associative}, i.e. that \(x \oplus (y \oplus z)=(x\oplus y)\oplus z\) for all \(x,y,z \in V\).

 \item  Show that there is no additive identity element, i.e. there is no element \(\mathbf{z} \in V\) such that \(x \oplus \mathbf{z} = x\) for all \(x \in V\).
 \end{enumerate}
 \end{problem}

 \begin{problem}{V1}
 Let \(V\) be the set of all pairs of real numbers with the operations, for any \((x_1,x_2), (y_1,y_2) \in V\), \(c\in \IR\),
 \begin{align*}
 (x_1,x_2) \oplus (y_1,y_2) &= (x_1+y_1,x_2y_2) \\
 c \odot (x_1,x_2) &= (cx_1, cx_2)
 \end{align*}
 \begin{enumerate}[(a)]
 \item Show that there is an \textbf{additive identity element}, i.e. an element \(\mathbf{z} \in V\) such that \((x_1,x_2)\oplus\vec{z}= (x_1,x_2)\) for any \( (x_1,x_2) \in V\).
\item Show that scalar multiplication does not distribute over vector addition, i.e. for some \(a \in \IR\) and \( (x_1,x_2), (y_1,y_2) \in V\), that \(a \odot \left( (x_1,x_2)\oplus (y_1,y_2) \right) \neq a \odot (x_1,x_2) \oplus a \odot (y_1,y_2) \).
 \end{enumerate}
\end{problem}
