%!TEX root =../../../course-notes.tex
% ^ leave for LaTeXTools build functionality

\begin{applicationActivities}

\begin{observation}
What happens when your tire hits a pothole?

\animategraphics[loop,width=\linewidth,controls=true]{15}{media/shocks/goodbadshocks-}{0}{88}


\end{observation}

\begin{activity}{5}
More abstractly, let's attach a mass (weighing \(m {\rm kg}\)) to a spring.

\begin{center}
\springmass
\end{center}
\vfill
List all forces acting on the mass.
\end{activity}

\begin{activity}{5}
\term{Hooke's law} says that the force exerted by the spring is proportional to the distance the spring is stretched.

\begin{center}
\springmass
\end{center}
\vfill
Write a differential equation modeling the displacement of the mass.
\end{activity}

\begin{observation}
There is an equillibrium point where the force of gravity balances the spring force.  If we measure displacement from this point, we can model the mass-spring system by
\[my''=ky.\]
\vfill
\begin{center}
\springmass
\end{center}
\end{observation}

\begin{activity}{15}
Consider the (numerically simplified) mass-spring equation \[y''=-y.\]
\begin{subactivity}
Find a solution.
\end{subactivity}
\begin{subactivity}
Find the general solution.
\end{subactivity}
\begin{subactivity}
Describe the long term behavior of the mass-spring system.
\end{subactivity}
\end{activity}

\begin{activity}{5}
In applications, this infinitely oscillating behavior is often inappropriate.
\vfill
Thus, a damper (dashpot) is often incorporated.  This provides a force proportional to the velocity.
\vfill
Write a differential equation modeling the displacement of a mass in a \textbf{damped} mass-spring system.
\vfill
\end{activity}

\begin{definition}
A \term{homogeneous second order constant coefficient} differential equation can be written in the form
\[ay''+by'+cy=0.\]
Here, \term{homogeneous} refers to the \(0\) on the right hand side of the equation.
\end{definition}

\begin{activity}{15}
Consider the second order constant coefficient equation \[y''=y.\]
\begin{subactivity}
Find a solution.
\end{subactivity}
\begin{subactivity}
Find the general solution.
\end{subactivity}
\begin{subactivity}
Describe the long term behavior of the solutions.
\end{subactivity}
\end{activity}




\end{applicationActivities}
