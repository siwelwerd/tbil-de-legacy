%!TEX root =../../../course-notes.tex
% ^ leave for LaTeXTools build functionality


\begin{applicationActivities}


\begin{observation}
What happens when your tire hits a pothole?

%\animategraphics[loop,width=\linewidth,controls=true]{15}{media/shocks/goodbadshocks-}{0}{88}
\url{https://prof.clontz.org/assets/img/good-bad-shocks.gif}

\end{observation}

%\begin{activity}{5}
%This motion may be modeled as a mass (weighing \(m\ {\rm kg}\)) hanging downward from a spring.
%
%\begin{center}
%\springmass
%\end{center}
%\vfill
%List all forces acting on the mass.
%\end{activity}

\begin{activity}{5}
\term{Hooke's law} says that the force exerted by the spring is proportional to the distance 
the spring is stretched from its natural length, given by a spring coefficient \(k>0\).

\begin{center}
\springmass
\end{center}
\vfill

Let \(y\) measure the displacement of the mass from the spring's natural length.
Write a differential equation modeling the displacement of the \(m\operatorname{kg}\) mass,
assuming that the only force acting on the mass comes from the spring.
\end{activity}

\begin{observation}
Since the spring acts on the mass in the opposite direction of displacement, we
may model the mass-spring system with
\[my''=-ky.\]
\vfill
\begin{center}
\springmass
\end{center}
\end{observation}

\begin{activity}{15}
Consider the mass-spring equation \(my''=-ky\) where \(m=k=1\): 
\[y''=-y.\]
\begin{subactivity}
Find a solution.
\end{subactivity}
\begin{subactivity}
Find the general solution.
\end{subactivity}
\begin{subactivity}
Describe the long term behavior of the mass-spring system.
\end{subactivity}
\end{activity}

\begin{activity}{5}
The general solution \(y=c_1\cos(t)+c_2\sin(t)\) models infinitely oscillating behavior,
but in applications this does not occur.
\vfill
Thus, a damper (a.k.a. dashpot) is often considered, which provides a force proportional to
velocity, given by the coefficient \(b>0\). For example, friction may act as a damper to a mass-spring system. 
\begin{center}
\springmassdamper
\end{center}
\vfill
Write a differential equation modeling the displacement of a mass in a \textbf{damped} mass-spring system.
\vfill
\end{activity}

\begin{observation}
The damped mass-spring system can be modelled by
\[my''=-by'-ky.\]
Here \(m\) is the mass, \(k\) is the spring constant, and \(b\) is the damping constant.  We can rearrange this as
\[y''+By'+Ky=0\]
where \(B=\frac{b}{m}\) and \(K=\frac{k}{m}\).
\vfill
This is a  \term{homogeneous second order constant coefficient} differential equation.
Here, \term{homogeneous} refers to the \(0\) on the right hand side of the equation.
\end{observation}

\begin{activity}{15}
Consider the second order constant coefficient equation \[y''=y.\]
\begin{subactivity}
Find a solution.
\end{subactivity}
\begin{subactivity}
Find the general solution.
\end{subactivity}
\begin{subactivity}
Describe the long term behavior of the solutions.
\end{subactivity}
\end{activity}

\begin{observation}
It is sometimes useful to think in terms of \term{differential  operators}.
\begin{itemize}
\item We will use \(D\) to represent a derivative.%; another common notation is \(\frac{\partial}{\partial x}\).  
  So for any function \(y\),  \[D(y)=\frac{\partial y}{\partial x}=y'.\]
\item \(D^2\) will denote the second derivative operator (i.e. differentiate twice, or apply D twice).
\item We will use \(I\) for the identity operator, so \(I(y)=y\).  
  (It can be thought of as \(I=D^0\), take the derivative zero times.)
\end{itemize}
\vfill
In this language, the differential equation \(y'+3y=0\) can be rewritten as \(D(y)+3I(y)=0\), 
or more simply \( (D+3I)(y)=0\).
\vfill
Thus, the question of solving the homogeneous differential equation is the question of finding the 
\term{kernel} of the differential operator \(D+3I\): all the functions \(y\) that the transformation
\(D+3I\) turns into the zero function.
\end{observation}

\begin{activity}{5}
Find a differential operator whose kernel is the solution set of the ODE \(y'=4y\).
\begin{enumerate}[a)]
\item \(D-4I\)
\item \(D+4I\)
\item \(D^2-4I\)
\item \(D^2+4D\)
\end{enumerate}
\end{activity}

\begin{activity}{5}
The kernel of the differential operator \(D-4I\) whose kernel is the general solution of the ODE 
\(y'=4y\). What is its general solution?
\begin{enumerate}[a)]
\item \(y=ke^{-4x}\)
\item \(y=ke^{4x}\)
\item \(y=4x+k\)
\item \(y=4\)
\end{enumerate}
\end{activity}

\begin{activity}{5}
What are ODE and general solution given by the kernel of the differential operator \(D-aI\) for a real number \(a\)?
\begin{enumerate}[a)]
\item \(y'-ay=0\) and \(y=ke^{ax}\).
\item \(y'+ay=0\) and \(y=ke^{-ax}\).
\item \(y'-a=0\) and \(y=ax+k\).
\item \(y''+a=0\) and \(y=-\frac{a}{2}x^2+kx+l\).
\end{enumerate}
\end{activity}

\begin{observation}
The kernel of the differential operator \(D-aI\) is given by the general solution \(y=ke^{ax}\).
\end{observation}

\begin{activity}{15}
Consider the ODE \[y''+5y'+6y=0.\]
\vfill
\begin{subactivity}
Use \(I,D,D^2\) to write a differential operator whose kernel is the solution set of the above ODE.
\end{subactivity}
\begin{subactivity}
Factor this differential operator as a composition of two simpler operators, as you would a polynomial. 
(This works because the order of applying the transformations \(D\) and \(I\) doesn't matter).
\end{subactivity}
\begin{subactivity}
Find the general solution for each factor, and then combine to find the general solution
to the overall ODE.
\end{subactivity}
\begin{subactivity}
Check that your general solution is valid by computing \(y',y''\) and plugging into
\(y''+5y'+6y=0\).
\end{subactivity}
\end{activity}

\begin{observation}
The kernel of \((D+3I)(D+2I)\) is given by \(y=k_1e^{-3t}+k_2e^{-2t}\).
\vfill
In general for \(\alpha\not=\beta\), the kernel of \((D-\alpha I)(D-\beta I)\) is given by
\(y=k_1e^{at}+k_2e^{bt}\).
\vfill
\end{observation}
%
%\begin{observation}
%If we let \(\L=D^2+5D+6I\), we can write the ODE \[y''+5y+6y=0\] as \[\L(y)=0.\]
%
%Note that such an \(\L\) is always a \term{linear transformation}.
%\end{observation}

%\begin{activity}{10}
%Consider the ODE \[y''+5y'-6y=0.\]
%\vfill
%\begin{subactivity}
%Find a differential operator whose kernel is the solution set of the above ODE.
%\end{subactivity}
%\begin{subactivity}
%Factor this differential operator as a composition of two simpler operators.
%(Note: 
%\end{subactivity}
%\begin{subactivity}
%Find the general solution of the ODE.
%\end{subactivity}
%\end{activity}

\begin{activity}{10}
Solve the ODE \[ 2y''+7y'+6y=0.\]
\end{activity}

\begin{activity}{15}
Recall that the general solution to \(y''+y=0\)
is given by \(y=c_1\sin(x)+c_2\cos(x)\).
Show how to find this solution using the differential
operator \(D^2+1\).
\end{activity}

\begin{activity}{15}
Consider the ODE \[y''+2y'+5y=0\].
\begin{subactivity}
Find its general solution using complex numbers.
\end{subactivity}
\begin{subactivity}
Describe the general solution only involving real numbers.
\end{subactivity}
\end{activity}

%\begin{observation}
%The kernel of \(aD^2+bD+cI\) is given by the general solution
%\(y=k_1e^{\alpha x}+k_2e^{\beta x}\) where \(\alpha\not=\beta\)
%are given by the roots of the polynomial \(ar^2+br+c\):
%\[\frac{-b\pm\sqrt{b^2-4ac}}{2a}\]
%When \(\alpha,\beta\) have imaginary parts, a substitution like
%\(k_1e^{ix}+k_2e^{-ix}=c_1\cos(x)+c_2\sin(x)\) must be used
%to obtain real-valued solutions.
%\end{observation}

\begin{activity}{5}
Which of these are solutions to the following ODE? 
\[ y''-4y'+4y=0\]
\begin{enumerate}[a)]
\item \(y=e^{2t}\), where \(y'=2e^{2t}\) and \(y''=4e^{2t}\)
\item \(y=te^{2t}\), where \(y'=e^{2t}+2te^{2t}\) and \(y''=4e^{2t}+4e^{2t}\)
\item \(y=e^{2t}+te^{2t}\), where \(y'=3e^{2t}+2te^{2t}\) and \(y''=8e^{2t}+4e^{2t}\)
\item All of the above
\end{enumerate}
\end{activity}

\begin{observation}
To solve \(y''-4y'+4y=0\), we need to find the kernel of \( (D-2I)(D-2I)=(D-2I)^2 \).
\begin{itemize}
\item The kernel of \(D-2I\) is given by \(ke^{2x}\).
\item But if \( (D-2I)(y) = e^{2t} \), then \((D-2I)(D-2I)(y)=(D-2I)(e^{2t})=0\) also.
\item That means the kernel of \((D-2I)^2\) is given by both
  \((D-2I)(y)=0\) and \((D-2I)(y)=e^{2t}\).
\end{itemize}
\end{observation}

\begin{activity}{15}
Solve \((D-2I)(y)=e^{2x}\).
\end{activity}


\begin{observation}
Since \((D-2I)(y)=0\) solves to \(ke^{2t}\) and \((D-2I)(y)=e^{2t}\) solves to
\(kte^{2t}\), 
we have shown that the general solution of \[y''-4y'+4y=0\] is \[y=c_0e^{2t}+c_1te^{2t}.\]
\end{observation}

\begin{activity}{10}
Consider the homogeneous second order constant coefficient ODE \[ay''+by'+cy=0.\]
If \(r\) is a number such that \(ar^2+br+c=0\), what can you conclude?
\begin{enumerate}[(a)]
\item \(e^{rt}\) is a solution.
\item \(e^{-rt}\) is a solution.
\item \(te^{rt}\) is a solution.
\item There are no solutions.
\end{enumerate}
\end{activity}

\begin{activity}{5}
Consider the homogeneous second order constant coefficient ODE \[ay''+by'+cy=0.\]

When does the general solution have the form \(c_0 e^{rt}+c_1 te^{rt}\) ?
\begin{enumerate}[(a)]
\item When the polynomial \(ax^2+bx+c\) has two distinct real roots.
\item When the polynomial \(ax^2+bx+c\) has a repeated real root.
\item When the polynomial \(ax^2+bx+c\) has two distinct non-real roots.
\item When the polynomial \(ax^2+bx+c\) has a repeated non-real root.
\end{enumerate}
\end{activity}


\begin{observation}
Consider the homogeneous second order constant coefficient ODE \[ay''+by'+cy=0\]
given by the differential operator \(aD^2+bD+cI\).
Let \(r\) be a (possibly non-real) solution to \(ax^2+bx+c=0\):
\vfill
\begin{itemize}
\item \(e^{rt}\) is a particular solution of the ODE.
\item If \(r\) is a double root, \(te^{rt}\) is also a particular solution.
\item if \(r=\alpha+\beta i\) is not real, Euler's formula allows us to express the real-valued solutions in terms of 
  \(\sin(\beta t)\) and \(\cos(\beta t)\).
\end{itemize}
Due to the usefulness of its solutions,
\(ax^2+bx+c=0\) is called the \term{auxiliary equation} for this ODE.
\end{observation}



\end{applicationActivities}
