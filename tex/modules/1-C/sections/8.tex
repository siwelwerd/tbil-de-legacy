%!TEX root =../../../course-notes.tex
% ^ leave for LaTeXTools build functionality

%30 minutes!
\begin{applicationActivities}

\begin{observation}
Consider the homogeneous second order constant coefficient ODE \[ay''+by'+cy=0.\]
\vfill
\begin{itemize}
\item If \(r\) is a root of \(ax^2+bx+c=0\), then \(e^{rt}\) is a solution of the ODE.
\item If \(r\) is a double root, variation of parameters shows that \(te^{rt}\) is also a solution.
\item if \(r\) is not real, Euler's formula allows us to express the solution in terms of \(\sin(rt)\) and \(\cos(rt)\).
\end{itemize}
\end{observation}

\begin{activity}{15}
Consider the following scenario:  a mass of \(4\ {\rm kg}\) suspended from a damped spring with spring constant \(k=2\ {\rm kg/s^2}\) and damping constant \(b= 6\ {\rm kg/s}\).  
\vfill
The mass is pulled down \(0.3\ {\rm m}\) and released from rest.  
\begin{subactivity}
Write down an ODE modelling this scenario, and find the general solution.
\end{subactivity}
\begin{subactivity}
Use the initial conditions \(y(0)=-0.3\) and \(y'(0)=0\) to find particular values of the constants.
\end{subactivity}
\end{activity}

\begin{definition}
In the previous problem, we needed to solve
\[ 4y''+6y'+2y = 0, \hspace{3em} y(0)=-0.3, \hspace{2em} y'(0)=0 .\]
This is called an \term{Initial Value Problem (IVP)} since we are provided with initial values of \(y\) and \(y'\).
\vfill
To solve an IVP, find a general solution of the ODE, and use the initial conditions to find the values of the constants.
\end{definition}

\begin{activity}{15}
Consider a mass of \(5\ {\rm kg}\) suspended from a damped spring with spring constant \(k=2\ {\rm kg/s^2}\) and damping constant \(b= 6{\rm kg/s}\).  
\vfill
The mass is pulled down \(0.3 {\rm m}\) and released from rest.  How many times does it pass back through its equillibrium state?
\vfill
\begin{enumerate}[(a)]
\item \(0\)
\item \(1\)
\item \(2\)
\item Infinitely many
\end{enumerate}
\end{activity}

\begin{observation}
It can be shown that in the \term{overdamped} situation, the spring might pass through the equillibrium position once (e.g. if given an initial push), but never more than once.
\end{observation}


\end{applicationActivities}
