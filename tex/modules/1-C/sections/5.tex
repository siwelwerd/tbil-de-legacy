%!TEX root =../../../course-notes.tex
% ^ leave for LaTeXTools build functionality

\begin{applicationActivities}

\begin{observation}
Consider the homogeneous second order constant coefficient ODE \[ay''+by'+cy=0.\]
\vfill
\begin{itemize}
\item If \(r\) is a root of \(ax^2+bx+c=0\), then \(e^{rt}\) is a solution of the ODE.
\item If \(r\) is a double root (that is, \(ax^2+bx+c=(x-r)^2\)), \(te^{rt}\) is also a solution.
\item If \(r=a+bi\) is not real, Euler's formula allows us to express
  \(e^{at+bit}\) in terms of \(e^{at}\), \(\sin(bt)\), and \(\cos(bt)\) to get
  a real-valued general solution.
\end{itemize}
\end{observation}

\begin{activity}{15}
Consider the following scenario:  a mass of \(4\ {\rm kg}\) suspended from a 
damped spring with spring constant \(k=2\ {\rm kg/s^2}\) and 
damping constant \(b= 6\ {\rm kg/s}\). As previously discussed, this is modeled
by the ODE
\[my''=-by'-ky.\]
\begin{subactivity}
Find the general solution for the ODE in terms of \(m,b,k\).
\end{subactivity}
\begin{subactivity}
The mass is pulled down \(0.3\ {\rm m}\) from its natural length and released from rest.  
Use the initial conditions \(y(0)=\unknown\) and \(y'(0)=\unknown\) to 
find the particular solution modeling this scenario.
\end{subactivity}
\end{activity}

\begin{activity}{5}
A \(1\ {\rm kg}\) mass is suspended from a spring with spring constant \(k= 9\ {\rm kg/s^2}\).
No damping is applied, but an external electromagnetic force of \(F(t)=\sin(t)\)
is applied. Which of these ODEs models this scenario?
\begin{enumerate}[a)]
\item \(my''+ky+\sin(t)=0\)
\item \(my''+ky=\sin(t)\)
\item \(my''+by'=\sin(t)\)
\item \(my''+by'+\sin(t)=0\)
\end{enumerate}
\end{activity}

\begin{observation}
Because \(my''\) is the total force acting on the object,
\(-by'-ky\) is the force acting on the object by the spring,
and an additional external force of \(F(t)\) is applied,
we get \(my''=-by'-ky+F(t)\) which rearranges to
\[my''+ky=\sin(t)\]
when \(b=0\) (no damping) and \(F(t)=\sin(t)\).

\vfill

This is an example of a \term{nonhomogeneous} second-order constant coefficient
equation of the form
\[
ay''+by'+cy=F(t)
\] 
since the \(F(t)=\sin(t)\) term is not a multiple of \(y\) or its derivatives.
As with first-order examples, these may be solved with variation of parameters.
\end{observation}

\begin{activity}{15}
Suppose \(y_1\) and \(y_2\) are two independent particular solutions of 
\(ay''+by'+cy=0\).  
\vfill
By variation of paraameters, we'll assume we can find a particular
solution \(y_p = v_1 y_1 + v_2 y_2\) for the ODE using
the currently unknown functions \(v_1,v_2\).
\vfill
\begin{subactivity}
Use the product rule (on \(v_1y_1\) and \(v_2y_2\)) to compute \(y_p'\).
\end{subactivity}

\begin{subactivity} 
Since we get to choose what \(v_1,v_2\) are, let's only look for examples
where \(v_1'y_1+v_2'y_2=0\) to simplify calculations.
Assuming this, compute \(y_p''\).
\end{subactivity}

\begin{subactivity}
Simplify the ODE \(ay_p''+by_p'+cy_p=f(x)\), keeping in mind that
\(ay_1''+by_1'+cy_1=0\) and \(ay_2''+by_2'+cy_2=0\).
\end{subactivity}
\vfill
\end{activity}

\begin{observation}
If we can find functions \(v_1\) and \(v_2\) that solve the
system of equations
\begin{alignat*}{2}
y_1 v_1' + y_2 v_2'&=0 \\
y_1' v_1' + y_2' v_2'&=\frac{1}{a}f(t)
\end{alignat*}
then \(y_p=y_1v_1+y_2v_2\) is a particular
solution for \(ay''+by'+cy=f(x)\).  
\end{observation}

\begin{activity}{20}
Consider the nonhomogeneous ODE \[y''+9y=\sin(t)\] of the
form \(ay''+by'+cy=f(t)\) for \(a=1,b=0,c=9,f(t)=\sin(t)\).
\vfill
\begin{subactivity}
Find \(y_h=k_1y_1+k_2y_2\), where \(y_1,y_2\) are
independent real-valued particular
solutions of \(y_h''+9y_h=0\).
\end{subactivity}
\begin{subactivity}
Substitute \(a,f(t),y_1,y_2,y_1',y_2'\) into
\begin{alignat*}{2}
y_1 v_1' + y_2 v_2'&=0 \\
y_1' v_1' + y_2' v_2'&=\frac{1}{a}f(t) 
\end{alignat*}
\end{subactivity}
\begin{subactivity}
Find \(v_1\), \(v_2\) by solving that system, and using
\(\int\sin(t)\cos(3t)dt=\frac{1}{8}\cos(t)\cos(3t)+\frac{3}{8}\sin(t)\sin(3t)+C\) and
\(\int\sin(t)\sin(3t)dt=-\frac{1}{8}\cos(t)\sin(3t)+\frac{3}{8}\sin(t)\cos(3t)+C\).
\end{subactivity}
\begin{subactivity}
Use \(y_p=y_1v_1+y_2v_2\) to write the general solution \(y=y_h+y_p\) 
of the original nonhomogeneous ODE.
\end{subactivity}
\end{activity}

\begin{activity}{10}
Consider the nonhomogeneous ODE \(y''+9y=\sin(3t)\).
\vfill
\begin{subactivity}
Find \(v_1\) and \(v_2\) by solving
\begin{alignat*}{2}
y_1 v_1' + y_2 v_2'&=0 \\
y_1' v_1' + y_2' v_2'&=\frac{1}{a}f(t) 
\end{alignat*}
for particular solutions \(y_1,y_2\) of \(y_h''+9y_h=0\).
Use \(\int\sin(3t)\cos(3t)dt=\frac{1}{6}\sin^2(3t)+C\) and
\(\int\sin^2(3t)dt=\frac{1}{6}(3t-\sin(3t)\cos(3t))+C\).
\end{subactivity}
\begin{subactivity}
Write the general solution of the original nonhomogeneous ODE.
\end{subactivity}
\end{activity}

\end{applicationActivities}
