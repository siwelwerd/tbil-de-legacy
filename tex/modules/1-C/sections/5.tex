%!TEX root =../../../course-notes.tex
% ^ leave for LaTeXTools build functionality

\begin{applicationActivities}

\begin{observation}
It is sometimes useful to think in terms of \term{differential  operators}.
\begin{itemize}
\item We will use \(D\) to represent a derivative; another common notation is \(\frac{\partial}{\partial x}\).  So for any function \(y\),  \[D(y)=\frac{\partial y}{\partial x}=y'.\]
\item \(D^2\) will denote the second derivative operator (i.e. differentiate twice, or apply D twice).
\item We will use \(I\) for the identity operator; it does nothing to a function.  That is, \(I(y)=y\).  It can be thought of as \(I=D^0\) (i.e. differentiate zero times).
\end{itemize}
\vfill
In this language, the differential equation \(y'+3y=0\) can be rewritten as \(D(y)+3I(y)=0\), or \( (D+3I)(y)=0\).
\vfill
Thus, the question of solving the homogeneous differential equation is the question of finding the \term{kernel} of the differential operator \(D+3I\).
\end{observation}

\begin{activity}{5}
What is the kernel of \(D-I\) ?
\vfill
\begin{subactivity}
Write a differential equation that corresponds to this question.
\end{subactivity}
\begin{subactivity}
Find the general solution of this differential equation.
\end{subactivity}
\end{activity}

\begin{activity}{5}
Find a differential operator whose kernel is the solution set of the ODE \(y'=4y\).
\end{activity}

\begin{activity}{10}
Consider the ODE \[y''+5y'+6y=0.\]
\vfill
\begin{subactivity}
Find a differential operator whose kernel is the solution set of the above ODE.
\end{subactivity}
\begin{subactivity}
Factor this differential operator as a composition of two operators. (This works because \(D\) and \(I\) commute).
\end{subactivity}
\begin{subactivity}
Find the general solution of the ODE.
\end{subactivity}
\end{activity}

\begin{observation}
If we let \(\L=D^2+5D+6I\), we can write the ODE \[y''+5y+6y=0\] as \[\L(y)=0.\]

Note that such an \(\L\) is always a \term{linear transformation}.
\end{observation}

\begin{activity}{5}
Find the general solution to
\[y''+y'-12y=0.\]
\end{activity}



\end{applicationActivities}
