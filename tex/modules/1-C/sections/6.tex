%!TEX root =../../../course-notes.tex
% ^ leave for LaTeXTools build functionality

\begin{applicationActivities}

<<<<<<< HEAD
=======
\begin{activity}{5}
Consider the ODE \[y''+5y-6y=0.\]
\vfill
\begin{subactivity}
Find a differential operator whose kernel is the solution set of the above ODE.
\end{subactivity}
\begin{subactivity}
Factor this differential operator as a composition of two operators. (This works because \(D\) and \(I\) commute).
\end{subactivity}
\begin{subactivity}
Find the general solution of the ODE.
\end{subactivity}
\end{activity}

\begin{activity}{5}
Solve the ODE \[ 2y''+7y'+6y=0.\]
\end{activity}

\begin{activity}{5}
An \term{Initial Value Problem (IVP)} consists of an ODE along with some initial conditions that allow you to determine a single solution.
\vfill
Solve the IVP \[2y''+7y'+6y=0, \hspace{3em} y(0)=1, \hspace{3em} y'(0)=0\].
\end{activity}

\begin{activity}{5}
Solve the ODE \[ y''+y=0.\]
\end{activity}

\begin{activity}{15}
Consider the ODE \[y''+2y'+5y=0\].
\vfill
\begin{subactivity}
Find the general solution.
\end{subactivity}
\begin{subactivity}
Describe the long-term behavior of the solutions.
\end{subactivity}
\end{activity}
>>>>>>> master

\begin{observation}
Solving \(y''+2y'+5y=0\) produced a general solution
\[y=c_1 e^{(-1+2i)t}+c_2 e^{(-1-2i)t}.\]
Applying Euler's formula yields
 \begin{alignat*}{2}
y&=c_1e^{-t} \left(\cos(2t)+i\sin(2t)\right)+c_2 e^{-t} \left(\cos(2t)-i\sin(2t)\right) \\
&=(c_1+c_2)e^{-t} \cos(2t) i(c_1-c_2) e^{-t} \sin(2t)
\end{alignat*}

which we can rewrite (letting \(k_1=c_1+c_2\) and \(k_2=i(c_1-c_2)\)) as
\[y=k_1 e^{-t} \cos(2t) + k_2 e^{-t}\sin(2t).\]
\end{observation}

\begin{activity}{15}
Solve the IVP
\[y''+6y'+34y=0, \hspace{3em} y(0)=2, \hspace{3em} y'(0)=4.\]
\end{activity}


\end{applicationActivities}
