%!TEX root =../../../course-notes.tex
% ^ leave for LaTeXTools build functionality

\begin{applicationActivities}

\begin{activity}{10}
Solve the ODE \[ y''-4y'+4y=0.\]
\end{activity}

\begin{observation}
To solve this, we need to find the kernel of \( (D-2I)(D-2I) \).
\begin{itemize}
\item The kernel of \(D-2I\) is \(\left\{ ce^{2t}\ \middle|\ c \in \IR \right\}\).
\item However, if \( (D-2I)(y) = Ae^{2t} \), then applying \(D-2I\) twice will yield zero!
\item So we must solve the ODE \[y'-2y=e^{2t}.\]
\end{itemize}
\end{observation}

\begin{activity}{15}
Solve \(y'-2y=e^{2t}\).
\end{activity}


\begin{observation}
Thus, we have shown that the general solution of \[y''-4y'+4y=0\] is \[y=c_0e^{2t}+c_1te^{2t}.\]
\end{observation}

\begin{activity}{15}
Solve \(y''-6y'+9y=0\).
\end{activity}

\begin{activity}{10}
Consider the homogeneous second order constant coefficient ODE \[ay''+by'+cy=0.\]
If \(r\) is a number such that \(ar^2+br+c=0\), what can you conclude?
\begin{enumerate}[(a)]
\item \(e^{rt}\) is a solution.
\item \(e^{-rt}\) is a solution.
\item \(te^{rt}\) is a solution.
\item There are no solutions.
\end{enumerate}
\end{activity}

\begin{activity}{5}
Consider the homogeneous second order constant coefficient ODE \[ay''+by'+cy=0.\]

When does the general solution have the form \(c_0 e^{rt}+te^{rt}\) ?
\begin{enumerate}[(a)]
\item When the polynomial \(ax^2+bx+c\) has two distinct real roots.
\item When the polynomial \(ax^2+bx+c\) has a repeated real root.
\item When the polynomial \(ax^2+bx+c\) has two distinct non-real roots.
\item When the polynomial \(ax^2+bx+c\) has a repeated non-real root.
\end{enumerate}
\end{activity}


\begin{observation}
Consider the homogeneous second order constant coefficient ODE \[ay''+by'+cy=0.\]
\vfill
\begin{itemize}
\item If \(r\) is a root of \(ax^2+bx+c=0\), then \(e^{rt}\) is a solution of the ODE.
\item If \(r\) is a double root, variation of parameters shows that \(te^{rt}\) is also a solution.
\item If \(r\) is not real, Euler's formula allows us to express the complex exponential part of the solution in terms of \(\sin(rt)\) and \(\cos(rt)\).
\end{itemize}
\end{observation}



\end{applicationActivities}
