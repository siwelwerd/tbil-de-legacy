%!TEX root =../../../course-notes.tex
% ^ leave for LaTeXTools build functionality

\begin{applicationActivities}

\begin{observation}
Recall the last activity from yesterday:
\vfill
Solve \(y'=y+2\)
\vfill
This is very similar to the equation \(y'=y\), which we know how to solve.
\end{observation}

\begin{activity}{15}
Solve \(y'=y+2\)
\vfill
\textbf{Simple idea:} Since \(e^t\) is a solution of \(y'=y\), we suppose a solution is of the form \(y_p = \mu e^t\) for some function \(\mu\). 
\begin{subactivity}
Subsitute \(y_p\) into the equation \(y'=y+2\) and simplify.
\end{subactivity}
\begin{subactivity}
Find \(\mu\).
\end{subactivity}
\begin{subactivity}
Find \(y_p\).
\end{subactivity}
\end{activity}

\begin{observation}
This technique is called \term{variation of parameters}.  If \(y_0\) is a solution of the \term{homogeneous} equation, we suppose a solution of the \term{non-homogeneous} equation has the form \(y_p = \mu y_0\), and then determine what \(\mu\) must be. 

\vfill

\textbf{Example: }
\begin{align*}
y'+3y = 0 & & \text{homogeneous} \\
y'+3y = x & & \text{non-homogeneous}
\end{align*}
\end{observation}

\begin{activity}{20}
Solve \(y'=x-3y\).
\begin{subactivity}
Solve the homogeneous equation \(y'+3y=0\).
\end{subactivity}
\begin{subactivity}
If \(y_0\) is a solution of the homogeneous equation, let \(y_p = \mu y_0 \) for some \textbf{function} \(\mu\).
Substitute \(y_p\) in to original equation and simplify.
\end{subactivity}
\begin{subactivity}
Determine \(\mu\), and then determine \(y_p\).
\end{subactivity}
\end{activity}

\begin{observation}
Since \(y_0=c_0 e^{-3x}\) was the general solution of the homogeneous equation, and \(y_p = \frac{x}{3}-\frac{1}{9}\) is a particular solution of the non-homogeneous equation, a general solution to the non-homegenous equation  
\[y'+3y=x\]
is
\[y_0+y_p = c_0e^{-3x}+\frac{x}{3}-\frac{1}{9}.\]

\end{observation}

\begin{activity}{15}
Find the general solution to \(y'=2y+x+1\).
\end{activity}





\end{applicationActivities}
