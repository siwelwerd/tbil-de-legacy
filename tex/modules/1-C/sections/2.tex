%!TEX root =../../../course-notes.tex
% ^ leave for LaTeXTools build functionality

\begin{applicationActivities}

\begin{observation}
Recall the last activity from yesterday:
\vfill
Solve \(y'=y+2\)
\vfill
This is very similar to the equation \(y'=y\), which we were able to solve.
\end{observation}

\begin{activity}{15}
Solve \(y'=y+2\)
\vfill
\textbf{Simple idea:} Since \(y_0=e^x\) was a particular solution of \(y'=y\), 
we guess that a particular solution for \(y'=y+2\) is of the form \(y_p =  v e^x\) 
for some \textbf{function} \(v(x)\). 
\begin{subactivity}
Use the Product Rule to find \(y_p'=\frac{d}{dx}[ve^x]\).
\end{subactivity}
\begin{subactivity}
Substitute \(y_p\) and \(y_p'\) into the equation \(y'=y+2\).
\end{subactivity}
\begin{subactivity}
Solve for \(v\).
\end{subactivity}
\begin{subactivity}
Find \(y_p\).
\end{subactivity}
\end{activity}

\begin{observation}
The technique outlined in the previous activity is called \term{variation of parameters}.  
If \(y_0\) is a particular solution of the \term{homogeneous} equation, 
assume that a particular solution of the 
\term{non-homogeneous} equation has the form \(y_p = v y_0\), and then determine what \(v\) must be. 

\vfill

\textbf{Example: }
\begin{align*}
y'+3y = 0 & & \text{homogeneous} \\
y'+3y = x & & \text{non-homogeneous}
\end{align*}

Note that each term of the homoegeneous equation includes \(y\) or it derivatives.
\end{observation}

\begin{activity}{20}
Solve \(y'=x-3y\) by first solving its corresopnding homogeneous equation and using
variation of parameters: 
\begin{align*}
y'+3y = 0 & & \text{homogeneous} \\
y'+3y = x & & \text{non-homogeneous}
\end{align*}
\begin{subactivity}
Modify \(e^x\) to find the general solution \(y_h\) for the homogeneous equation.
\end{subactivity}
\begin{subactivity}
Choose a particular solution \(y_0\) for the homoegeneous equation, and
assume \(y_p = v y_0 \) is a particular solution of the non-homogeneous equation for some \textbf{function} \(v\).
Substitute \(y_p\) into non-homogeneous equation and simplify.
\end{subactivity}
\begin{subactivity}
Determine \(v\), and then determine \(y_p\).
\end{subactivity}
\end{activity}

\begin{observation}
Since \(y_h=ke^{-3x}\) was the general solution of \(y'+3y=0\), 
and \(y_p = \frac{x}{3}-\frac{1}{9}\) is a particular solution of \(y'+3y=x\), 
\[y=y_h+y_p = \left(ke^{-3x}\right)+\left(\frac{x}{3}-\frac{1}{9}\right)\]
is a solution to \(y'+3y=x\):

\vfill

\[\frac{d}{dx}[y_h+y_p]+3(y_h+y_p)=(y_h'+3y_h)+(y_p'+3y_p)=0+x=x\]
\end{observation}

\begin{fact}
Let \(a\) be a constant real number.
Every constant coefficient first order ODE
\[y'+ay=f(x)\]
has the general solution
\[y=y_h+y_p\]
where \(y_h\) is the general solution to the homogeneous equation \(y'+ay=0\)
and \(y_p\) is a particular solution to \(y'+ay=f(t)\).
\end{fact}

\begin{activity}{15}
Find the general solution to \(y'=2y+x+1\) using variation of parameters:

\vfill

\begin{itemize}
\item Write the homoegeneous equation and find its general solution \(y_h\).
\item Use a particular solution \(y_0\) for the homogeneous equation to find a particular solution
      \(y_p=vy_0\) for the original equation.
\item Let \(y=y_h+y_p\) give your general solution to the equation.
\end{itemize}
\end{activity}





\end{applicationActivities}
