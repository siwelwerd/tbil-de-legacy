\begin{applicationActivities}

\begin{observation}
In this module, we want to learn how to model (and solve) situtations with \term{discontinuous} force, such as
\begin{itemize}
	\item Collisions
	\item Thrust that can be turned on and off instantly
	\item Applied voltages that can be turned on and off instantly
\end{itemize}
\vfill
Today we will learn how to model these forces, and introduce a tool called them \term{Laplace Transform} that we will use to solve the resulting IVPs.
\end{observation}


\begin{activity}{10}
A \(1\ {\rm kg}\) mass is hung from a spring with spring constant \(k=1\ {\rm N/m}\).  The mass is at rest, when it is hit with a hammer imparting \(3 {\rm N s}\) of upward impulse.
\vfill
\begin{columns}
\begin{column}{0.3\textwidth}
\begin{center}
\springmass
\end{center}
\end{column}
\begin{column}{0.7\textwidth}
\begin{subactivity}
Draw a graph of the kinetic energy in the system with respect to time.
\end{subactivity}
\begin{subactivity}
Write an initial value problem modelling this system.
\end{subactivity}
\end{column}
\end{columns}
\end{activity}

\begin{definition}
The \term{Dirac delta distribution} \(\delta(t)\) models the application of instantaneous force.  \textbf{It is not a function}, but makes sense in definite integrals: 
\vfill
If \(a,b\) is any open interval containing \(0\), then
\[ \int _a ^b f(t)\delta(t)dt = f(0)\]
for any function \(f(t)\) that is continuous around \(0\).
\end{definition}

\begin{definition}
The \term{unit impulse function} \(u(t)\) is given by \[u(t)=\begin{cases} 1 & t > 0 \\ 0 & t<0. \end{cases}\]
\vfill
Note that \(u(s)=\int _{-\infty} ^s  \delta(t)\ dt\); in this fuzzy sense, \(\delta\) is the derivative of \(u(t)\) (which is not differentiable everywhere!)
\end{definition}

\begin{activity}{10}
Try to solve the IVP 
\[y''+y=\delta(t)\]

\vfill
Where does our existing technique break down?
\end{activity}

\begin{observation}
To get around this difficulty, we will apply an \term{integral transform} called the \term{Laplace Transform} to our ODE.
\vfill
\begin{itemize}
\item We want to use a definite integral to handle things like \(\delta\), which we can only understand via a definite integral.
\item Since we are focused on IVPs, we can integrate starting at \(0\), but need to go to \(\infty\)
\item But now we need to worry about convergence--thus we will multiply by a suitable function that decays fast enough to make most functions converge.
\end{itemize}
\end{observation}

\begin{activity}{5}
Arrange the following functions in order of how fast they decay to zero in the limit at infinity:
\begin{enumerate}[(A)]
\item \(x^{-n}\) for a positive integer \(n\)
\item \(e^{-ax}\) for a positive integer \(a\)
\item \(\frac{1}{\ln(ax)}\) for a positive integer \(a\)
\item \(\frac{1}{\ln(x^n)}\) for a positive integer \(n\)
\end{enumerate}
\end{activity}

\begin{definition}
The \term{Laplace Transform} of a function \(f(t)\) is the function
\[\L\{f\}(s)=\int _0 ^\infty e^{-st}f(t)\ dt.\]
\vfill
Note that the Laplace transform turns a function of \(t\) into a function of \(s\).
\vfill
Moreover, \(\L\) is linear: \(\L\{f+g\}=\L\{f\}+\L\{g\}\), and \(\L\{cf\}=c\L\{f\}\) for constants \(c\).
\end{definition}


\begin{activity}{5}
Recall that \[\L\{f\}(s)=\int _0 ^\infty e^{-st} f(t)\ dt.\]
\begin{subactivity}
Compute \(\L(\delta(t))\)
\end{subactivity}
\begin{subactivity}
If \(a>0\), compute \(\L\{\delta(t-a)\}\)
\end{subactivity}
\end{activity}

\begin{activity}{5}
Recall that \[\L\{f\}(s)=\int _0 ^\infty e^{-st} f(t)\ dt.\]
\begin{subactivity}
Compute \(\L\{e^t\}\)
\end{subactivity}
\begin{subactivity}
If \(a>0\), compute \(\L\{e^{at}\}\)
\end{subactivity}
\end{activity}

\begin{activity}{15}
Recall that \[\L\{f\}(s)=\int _0 ^\infty e^{-st} f(t)\ dt.\]
\begin{subactivity}
Compute \(\L\{1\}\)
\end{subactivity}
\begin{subactivity}
Compute \(\L\{t\}\)
\end{subactivity}
\begin{subactivity}
Compute \(\L\{t^2\}\)
\end{subactivity}
\begin{subactivity}
Compute \(\L\{t^3\}\)
\end{subactivity}
\begin{subactivity}
Compute \(\L\{t^4\}\)
\end{subactivity}
\end{activity}





\end{applicationActivities}
