\begin{problem}{S3}
\begin{minipage}[t]{0.8\linewidth}
Consider the dual mass-spring system at the right.  Both masses are \(1 {\rm kg}\).  The upper spring has a spring constant of \( 1 {\rm N/m}\) while the lower spring constant is \(6 {\rm N/m}\).  The lower spring is pulled down \(1 {\rm m}\) without disturbing the upper spring, and released from rest. 
\begin{enumerate}[(a)]
\item Write down an IVP modelling the motion of the two masses.
\item Where will the two masses be after \(2 {\rm s}\)?
\end{enumerate}
\end{minipage}
\hfill
\springdoublemassQuiz[0.7]
\hfill
\end{problem}

\begin{problem}{S3}
\begin{minipage}[t]{0.8\linewidth}
Consider the dual mass-spring system at the right.  Both masses are \(1 {\rm kg}\).  The upper spring has a spring constant of \( 1 {\rm N/m}\) while the lower spring constant is \(6 {\rm N/m}\).  The lower spring is pulled down \(0.5 {\rm m}\) without disturbing the upper spring, and released from rest. 
\begin{enumerate}[(a)]
\item Write down an IVP modelling the motion of the two masses.
\item Where will the two masses be after \(3 {\rm s}\)?
\end{enumerate}
\end{minipage}
\hfill
\springdoublemassQuiz[0.7]
\hfill
\end{problem}

\begin{problem}{S3}
\begin{minipage}[t]{0.8\linewidth}
Consider the dual mass-spring system at the right.  Both masses are \(1 {\rm kg}\).  The upper spring has a spring constant of \( 1 {\rm N/m}\) while the lower spring constant is \(6 {\rm N/m}\).  The upper spring is pushed up \(0.5 {\rm m}\) without disturbing the upper spring, and released from rest. 
\begin{enumerate}[(a)]
\item Write down an IVP modelling the motion of the two masses.
\item Where will the two masses be after \(4 {\rm s}\)?
\end{enumerate}
\end{minipage}
\hfill
\springdoublemassQuiz[0.7]
\hfill
\end{problem}

\begin{problem}{S3}
\begin{minipage}[t]{0.8\linewidth}
Consider the dual mass-spring system at the right.  Both masses are \(3 {\rm kg}\).  The upper spring has a spring constant of \( 8 {\rm N/m}\) while the lower spring constant is \(6 {\rm N/m}\).  The upper spring is pushed up \(0.5 {\rm m}\) without disturbing the lower spring, and released from rest. 
\begin{enumerate}[(a)]
\item Write down an IVP modelling the motion of the two masses.
\item Where will the two masses be after \(2 {\rm s}\)?
\end{enumerate}
\end{minipage}
\hfill
\springdoublemassQuiz[0.7]
\hfill
\end{problem}

\begin{problem}{S3}
\begin{minipage}[t]{0.8\linewidth}
Consider the dual mass-spring system at the right.  Both masses are \(3 {\rm kg}\).  The upper spring has a spring constant of \( 8 {\rm N/m}\) while the lower spring constant is \(6 {\rm N/m}\).  The lower spring is pulled down \(0.5 {\rm m}\) without disturbing the upper spring, and released from rest. 
\begin{enumerate}[(a)]
\item Write down an IVP modelling the motion of the two masses.
\item Where will the two masses be after \(2 {\rm s}\)?
\end{enumerate}
\end{minipage}
\hfill
\springdoublemassQuiz[0.7]
\hfill
\end{problem}

\begin{problem}{S3}
\begin{minipage}[t]{0.8\linewidth}
Consider the dual mass-spring system at the right.  The upper mass is \(2 {\rm kg}\) while the lower mass is \(1 {\rm kg}\).  The upper spring has a spring constant of \( 6 {\rm N/m}\) while the lower spring constant is \(4 {\rm N/m}\).  The lower spring is pulled down \(0.5 {\rm m}\) without disturbing the upper spring, and released from rest. 
\begin{enumerate}[(a)]
\item Write down an IVP modelling the motion of the two masses.
\item Where will the two masses be after \(3 {\rm s}\)?
\end{enumerate}
\end{minipage}
\hfill
\springdoublemassQuiz[0.7]
\hfill
\end{problem}

\begin{problem}{S3}
\begin{minipage}[t]{0.8\linewidth}
Consider the dual mass-spring system at the right.  The upper mass is \(2 {\rm kg}\) while the lower mass is \(1 {\rm kg}\).  The upper spring has a spring constant of \( 6 {\rm N/m}\) while the lower spring constant is \(4 {\rm N/m}\).  The lower spring is pulled down \(1 {\rm m}\) without disturbing the upper spring, and released from rest. 
\begin{enumerate}[(a)]
\item Write down an IVP modelling the motion of the two masses.
\item Where will the two masses be after \(2 {\rm s}\)?
\end{enumerate}
\end{minipage}
\hfill
\springdoublemassQuiz[0.7]
\hfill
\end{problem}

\begin{problem}{S3}
\begin{minipage}[t]{0.8\linewidth}
Consider the dual mass-spring system at the right.  The upper mass is \(1 {\rm kg}\) while the lower mass is \(2 {\rm kg}\).  The upper spring has a spring constant of \( 3 {\rm N/m}\) while the lower spring constant is \(4 {\rm N/m}\).  The lower spring is pulled down \(0.5 {\rm m}\) without disturbing the upper spring, and released from rest. 
\begin{enumerate}[(a)]
\item Write down an IVP modelling the motion of the two masses.
\item Where will the two masses be after \(3 {\rm s}\)?
\end{enumerate}
\end{minipage}
\hfill
\springdoublemassQuiz[0.7]
\hfill
\end{problem}

\begin{problem}{S3}
\begin{minipage}[t]{0.8\linewidth}
Consider the dual mass-spring system at the right.  The upper mass is \(1 {\rm kg}\) while the lower mass is \(2 {\rm kg}\).  The upper spring has a spring constant of \( 3 {\rm N/m}\) while the lower spring constant is \(4 {\rm N/m}\).  The lower spring is pulled down \(1 {\rm m}\) without disturbing the upper spring, and released from rest. 
\begin{enumerate}[(a)]
\item Write down an IVP modelling the motion of the two masses.
\item Where will the two masses be after \(2 {\rm s}\)?
\end{enumerate}
\end{minipage}
\hfill
\springdoublemassQuiz[0.7]
\hfill
\end{problem}
