\begin{applicationActivities}

\begin{observation}
We previously saw that the first order IVP
\[ y'=f(x,y), \hspace{5em} y(x_0)=y_0 \]
had a unique solution on some (possibly tiny!) interval containing \(x_0\) when \(f(x,y)\) and \(\frac{\partial f}{\partial y}\) are both continuous at \((x_0,y_0)\).
\vfill
\end{observation}

\begin{activity}{10}
Consider the second order ODE 
\[(x^2-1)^2y''+4x=0\].
\begin{subactivity}
Solve for \(y''\), and then integrate to find \(y'\).
\end{subactivity}
\begin{subactivity}
Integrate again to find \(y\). ({\bf Hint: } \(\frac{2}{x^2-1}=\frac{1}{x-1}-\frac{1}{x+1}\))
\end{subactivity}
\begin{subactivity}
For what values of \(x\) is your solution valid?
\end{subactivity}
\end{activity}

\begin{observation}
The ODE \( (x^2-1)^2y''+4x=0\) did not have a solution where the coefficient of \(y''\) vanished, i.e. at \(x=1\) and \(x=-1\).
\vfill
In general, if \(x_1 < x < x_2\) is an interval containing \(x_0\) for which
\begin{itemize}
\item \(a(x),b(x),c(x),\text{ and } f(x)\) are continuous, and
\item \( a(x)\) does not vanish
\end{itemize}
Then the second order {\bf linear} IVP
\[ a(x)y''+b(x)y'+c(x)y=f(x), \hspace{5em} y(x_0)=y_0,\ y'(x_0)=y_1 \]
will have a unique solution on \(x_1<x<x_2\).
\end{observation}

\begin{observation}
Our uniqueness result for first order equations applied to all first order equations.  Our second order result applies to only {\bf linear} equations, but provides added information--a precise interval on which the unique solution exists.
\vfill
For example, the IVP 
\[ (x^2-1)^2y''+4x=0 , \hspace{5em} y(2)=3,\ y'(2)=4\]
will have a unique solution valid for \(1<x<\infty\).
\end{observation}


\begin{activity}{5}
Consider the IVP
\[ \sin(x)y''+\cos(x)y= x^2-4, \hspace{5em} y\left(\frac{\pi}{4}\right)=1,\ y'\left(\frac{\pi}{4}\right)=0.\]
Determine the largest interval on which a unique solution is guaranteed to exist.
\end{activity}

\begin{activity}{5}
Consider the IVP
\[ y''+\frac{1}{x}y'-\frac{1}{x-4}y=0 \hspace{5em} y\left(2\right)=5,\ y'\left(2\right)=-1.\]
Determine the largest interval on which a unique solution is guaranteed to exist.
\end{activity}


\begin{activity}{5}
Consider the ODE
\[ (x^2-1)y'' + \frac{1}{x}y' + e^x y = 0.\]
Determine \textbf{all} intervals on which a unique solution is guaranteed to exist.
\end{activity}


\begin{activity}{5}
Consider the ODE
\[ \frac{x}{x-1}y'' + \frac{x+2}{x+1}y' + e^{-x} y = 0.\]
Determine \textbf{all} intervals on which a unique solution is guaranteed to exist.
\end{activity}


\begin{activity}{5}
Consider the ODE
\[ \sqrt{x^2-1} y'' + y' + \frac{1}{x} y = 0.\]
Determine \textbf{all} intervals on which a unique solution is guaranteed to exist.
\end{activity}




\end{applicationActivities}
