\begin{applicationActivities}

\begin{observation}
The same problem we saw with a first order ODE failing to have a unique solution can also occur in systems of first order ODEs.

\vfill

Thus, to ensure that the system
\begin{align*}
x' &= f(t,x,y) \\
y' &= g(t,x,y) 
\end{align*}
has a unique solution, you must check that \( f(t,x,y), \frac{\partial f}{\partial x}, \frac{\partial f}{\partial y}, g(t,x,y), \frac{\partial g}{\partial x}, \text{and} \frac{\partial g}{\partial y}\) are all continuous near the initial point.
\end{observation}

\begin{observation}
If the system is linear, we can say more!
\vfill
Suppose \( a(t), b(t), c(t), d(t), f(t), g(t)\) are all continous on the interval \( h<t<k\).  Then for any \(h<t_0<k\), the IVP
\begin{align*}
x' &= a(t)x+b(t)y + f(t)  & x(t_0) &=  x_0\\
y' &= c(t)x+d(t)y + g(t) & y(t_0) &= y_0 
\end{align*}
has a unique solution on the (time) interval \(h<t<k\).
\end{observation}


\begin{activity}{5}
Consider the IVP
\begin{align*}
x' &= \frac{1}{t-1} x + \sqrt{t} y + t^2  & x(2)=5 \\
y' &= \frac{1}{t} x + \sqrt{t+1} y + t^2  & y(2)=7 
\end{align*}
\vfill
What is the largest interval on which this IVP has a unique solution?
\end{activity}

\begin{activity}{5}
Determine \textbf{all intervals} on which a unique solution is guaranteed to exist for the below system.
\begin{align*}
x' &= \frac{1}{t-1} x + \sqrt{t} y + t^2  \\
y' &= \frac{1}{t} x + \sqrt{t+1} y + t^2 
\end{align*}

\end{activity}

\begin{activity}{5}
Determine \textbf{all intervals} on which a unique solution is guaranteed to exist for the below system.
\begin{align*}
x' &= \ln(t-2) x + \sqrt{t} y + \frac{1}{t-1}  \\
y' &= \cos(t) x +  y  
\end{align*}

\end{activity}


\begin{activity}{10}
Euler's method can be extended to systems in a straightforward way.
\vfill

Consider the system IVP
\begin{align*}
x' &= 3x+4y-t & x(1)&=2 \\
y' &= x-y+t & y(1)&=3 .
\end{align*}

\begin{subactivity}
Compute \(x'\) and \(y'\)  when \(t=1\).
\end{subactivity}
\begin{subactivity}
Use linear approximations to estimate the values of \(x(1.1)\) and \(y(1.1)\).
\end{subactivity}
\begin{subactivity}
Calculate the slopes \(x'\) and \(y'\) when \(t=1.1\) (and as just calculated, \(x(1.1)=\) and \(y(1.1)=\)). 
\end{subactivity}
\begin{subactivity}
Use linear approximations to estimate the values of \(x(1.2)\) and \(y(1.2)\).
\end{subactivity}
\end{activity}

\begin{observation}
\begin{align*}
x' &= 3x+4y-t & x(1)&=2 \\
y' &= x-y+t & y(1)&=3 .
\end{align*}
\vfill
It is often convenient to organize this information in a table
\begin{center}
\begin{tabular}{c|c|c||c|c||c|c|c|}
\(t_n\) & \(x_n\) & \(y_n\) & \(x'(t_n,x_n,y_n)\) &  \(y'(t_n,x_n,y_n)\) & \(t_{n+1}\) & \(x_{n+1}\) & \(y_{n+1}\) \\ \hline \hline
1 & 2 & 3 & 17 & 0 & 1.1 & 2.17 & 3  \\ \hline
1.1 & 2.17 & 3 & 17.41 & 0.27 & 1.2 & 3.911 & 3.027 \\ \hline
1.2 & 3.911 & 3.027 &  &  &  &  &  \\ \hline
\end{tabular}
\end{center}
\vfill
Thus \(x(1.2) \approx 3.911\) and \(y(1.2) \approx 3.027\).
\end{observation}

\begin{activity}{10}
Use Euler's method to estimate \(x(3.3)\) and \(y(3.3)\).
\begin{align*}
x' &= 3x-ty & x(3)&=2 \\
y' &= x-y^2 & y(3)&=1 .
\end{align*}
\end{activity}

\begin{activity}{10}
Use Euler's method to estimate \(x(4.6)\) and \(y(4.6)\).
\begin{align*}
x' &= xy-t & x(4)&=2 \\
y' &= x+t & y(4)&=0 .
\end{align*}
\end{activity}


\end{applicationActivities}
