\begin{problem}{A1}
Consider the following maps of polynomials \(S: \P^6 \rightarrow \P^6\)
and \(T:\P^6\rightarrow\P^6\) defined by
\[S(f(x))= f(x)+3 \text{ and }T(f(x)) = f(x)+f(3).\]
Show that one of these maps is a linear transformation, and that the other
map is not.
\end{problem}
\begin{solution}
  \(T\) is linear, \(S\) is not.
\end{solution}

\begin{problem}{A1}
Consider the following maps of polynomials \(S: \P^4 \rightarrow \P^5\)
and \(T:\P^4\rightarrow\P^5\) defined by
\[S(f(x))= xf(x)-f(1) \text{ and }T(f(x)) = xf(x)-x.\]
Show that one of these maps is a linear transformation, and that the other
map is not.
\end{problem}
\begin{solution}
  \(S\) is linear, \(T\) is not.
\end{solution}

\begin{problem}{A1}
Consider the following maps of polynomials \(S: \P \rightarrow \P\)
and \(T:\P\rightarrow\P\) defined by
\[S(f(x))= f'(x)-f''(x) \text{ and }T(f(x)) = f(x)-(f(x))^2.\]
Show that one of these maps is a linear transformation, and that the other
map is not.
\end{problem}
\begin{solution}
  \(S\) is linear, \(T\) is not.
\end{solution}


\begin{problem}{A1}
Consider the following maps of polynomials \(S: \P^2 \rightarrow \P^4\)
and \(T:\P^2\rightarrow\P^4\) defined by
\[S(f(x))= x^2f(x) \text{ and }T(f(x)) = (f(x))^2.\]
Show that one of these maps is a linear transformation, and that the other
map is not.
\end{problem}
\begin{solution}
  \(S\) is linear, \(T\) is not.
\end{solution}

\begin{problem}{A1}
Consider the following maps of polynomials \(S: \P \rightarrow \P\)
and \(T:\P\rightarrow\P\) defined by
\[S(f(x))= (f(x))^2+1 \text{ and }T(f(x)) = (x^2+1)f(x).\]
Show that one of these maps is a linear transformation, and that the other
map is not.
\end{problem}
\begin{solution}
  \(T\) is linear, \(S\) is not.
\end{solution}

\begin{problem}{A1}
Consider the following maps of polynomials \(S: \P^2 \rightarrow \P^2\)
and \(T:\P^2\rightarrow\P^2\) defined by
\[S(ax^2+bx+c)= cx^2+bx+a \text{ and }T(ax^2+bx+c) = a^2x^2+b^2x+c^2.\]
Show that one of these maps is a linear transformation, and that the other
map is not.
\end{problem}
\begin{solution}
  \(S\) is linear, \(T\) is not.
\end{solution}

\begin{problem}{A1}
Consider the following maps of polynomials \(S: \P^2 \rightarrow \P^1\)
and \(T:\P^2\rightarrow\P^1\) defined by
\[S(ax^2+bx+c)= 2ax+b \text{ and }T(ax^2+bx+c) = a^2x+b.\]
Show that one of these maps is a linear transformation, and that the other
map is not.
\end{problem}
\begin{solution}
  \(S\) is linear, \(T\) is not.
\end{solution}

\begin{problem}{A1}
Consider the following maps of polynomials \(S: \P^2 \rightarrow \P^3\)
and \(T:\P^2\rightarrow\P^3\) defined by
\[S(ax^2+bx+c)= ax^3+bx^2+cx \text{ and }T(ax^2+bx+c) = abcx^3.\]
Show that one of these maps is a linear transformation, and that the other
map is not.
\end{problem}
\begin{solution}
  \(S\) is linear, \(T\) is not.
\end{solution}
