\begin{readinessAssuranceTest}
\setcounter{enumi}{10}

%C
\item Which of the following sets describes where the polynomial \(f(x)=x^3(x-1)^2(x+1)\) is \textbf{negative}?
\begin{multicols}{4}
\begin{readinessAssuranceTestChoices}
\item \((-1,0) \cup (1,\infty) \)
\item \((-\infty,-1) \cup (0,1) \)
\item \((-1,0)\) %Correct
\item \((0,1)\)
\end{readinessAssuranceTestChoices}
\end{multicols}
\vfill

%C
\item
Compute \( \int \frac{4}{4-x}\ dx \).
\begin{multicols}{4}
\begin{readinessAssuranceTestChoices}
\item \(\ln|4-x|+C\)
\item \(-\ln|4-x|+C\)
\item \(-4\ln|4-x|+C\) %Correct
\item \(4\ln|4-x|+C\)
\end{readinessAssuranceTestChoices}
\end{multicols}

\vfill
%A
\item
Compute \(\int e^{2x+1}\ dx\)
\begin{multicols}{4}
\begin{readinessAssuranceTestChoices}
\item \( \frac{1}{2} e^{2x+1}+C\) %Correct
\item \(2e^{2x+1}+C \)
\item \(e^{2x+1}+C\)
\item \( (2x+1)e^{2x+1}+C\)
\end{readinessAssuranceTestChoices}
\end{multicols}

\vfill
%C
\item Compute \(\int x e^{x^2+1}\ dx\).
\begin{multicols}{4}
\begin{readinessAssuranceTestChoices}
\item \(e^{x^2+1}+C\)
\item \(xe^{x^2+1}+C\)
\item \(\frac{1}{2} e^{x^2+1}+C\) %Correct
\item \(2 e^{x^2+1}+C\) 
\end{readinessAssuranceTestChoices}
\end{multicols}


\vfill
%A
\item Compute \(\int x^2e^x\ dx\).
\begin{multicols}{4}
\begin{readinessAssuranceTestChoices}
\item \((x^2-2x+2)e^x+C\) %Correct
\item \(x^2e^x+C\)
\item \( 2xe^{x}+C\)
\item \( (x^2-2)e^x+C\)
\end{readinessAssuranceTestChoices}
\end{multicols}

\vfill
\newpage
%C
\item
Compute \(\int \frac{4}{4-x^2}\ dx\).
\begin{multicols}{4}
\begin{readinessAssuranceTestChoices}
\item \( \ln |4-x^2|+C\) 
\item \( \ln |x^2-4|+C\)
\item \( \ln \left|\frac{2+x}{2-x}\right|+C\) %Correct
\item \( \ln \left|\frac{x-2}{x+2}\right|+C\)
\end{readinessAssuranceTestChoices}
\end{multicols}


\vfill
%B
\item
Compute \( \int x\sin(x)\ dx \).
\begin{multicols}{4}
\begin{readinessAssuranceTestChoices}
\item \(-x\cos(x)+C\)
\item \( \sin(x)-x\cos(x)+C \) %Correct
\item \(\frac{1}{2}x^2 \cos(x)+C \)
\item \( \frac{1}{2}x^2-\cos(x)+C\)
\end{readinessAssuranceTestChoices}
\end{multicols}

\vfill

%B
\item Exactly one of the four vector fields below is conservative.  Identify which one is conservative.
\begin{multicols}{4}
\begin{readinessAssuranceTestChoices}
\item \(\langle 2xy,y^2\rangle\)
\item \(\langle y^2,2xy\rangle\) %Correct
\item \(\langle x^2,2xy\rangle\)
\item \(\langle 2xy,2xy\rangle\)
\end{readinessAssuranceTestChoices}
\end{multicols}

\vfill
%B
\item
Which of the following is a potential function for the vector field \( \langle 2xy+2, x^2-3y^2 \rangle \).
\begin{multicols}{4}
\begin{readinessAssuranceTestChoices}
\item \( xy^2+2y+\frac{1}{3}x^3-3xy^2\) 
\item \( x^2y-y^3+2x+3\) %Correct
\item \( x^2y+2x+3\) 
\item \( x^2y-y^3\) 
\end{readinessAssuranceTestChoices}
\end{multicols}


\vfill

%C
\item Find the general solution to \(y'-y=3-x\).
\begin{multicols}{4}
\begin{readinessAssuranceTestChoices}
\item \(y=ke^{-x}+x-1\) 
\item \(y=ke^{-x}+2x-3\) 
\item \(y=ke^x+x-2\) %Correct
\item \(y=ke^x+3-x\) 
\end{readinessAssuranceTestChoices}
\end{multicols}

\vfill

\end{readinessAssuranceTest}
