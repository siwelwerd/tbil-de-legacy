%!TEX root =../../../course-notes.tex
% ^ leave for LaTeXTools build functionality
\begin{applicationActivities}

%30 minutes
\begin{observation}
There are two very simple kinds of separable ODEs.
\vfill
Equations of the form \(y'=f(x)\) can be solved immediately by integrating and produce explicit solutions.
\vfill
Equations of the form \(y'=f(y)\) are often impossible or difficult to solve explicitly.  They are called \term{autonomous} equations.
\end{observation}

\begin{activity}{10}
Consider the autonomous equation \[y'=y^2\].
\vfill
Suppose a solution goes through the point \(y(10)=50 \).  What can you say about \(y(11)\)?
\vfill
\begin{enumerate}[(a)]
\item \(y(10)<y(11)\)
\item \(y(10)=y(11)\)
\item \(y(10)>y(11)\)
\end{enumerate}
\end{activity}

\begin{activity}{10}
Consider the autonomous equation \[y'=y^2(y-2).\]

\begin{subactivity}
Draw a number line for \(y'\), indicating where it is positive or negative.
\end{subactivity}
\begin{subactivity}
What can you say about the long term behavior of a solution passing through \(y(4)=1\)?
\end{subactivity}
\begin{subactivity}
What can you say about the long term behavior of a solution passing through \(y(2)=0.001\)?
\end{subactivity}
\begin{subactivity}
What can you say about the long term behavior of a solution passing through \(y(2)=-0.001\)?
\end{subactivity}
\end{activity}

\end{applicationActivities}
