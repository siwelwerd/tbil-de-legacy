%!TEX root =../../../course-notes.tex
% ^ leave for LaTeXTools build functionality
\begin{applicationActivities}

\begin{observation}
Consider a (massless) particle in a fluid flow.


What path does the particle take?
\end{observation}

\begin{observation}
A vector field \(\langle P,Q\rangle\) corresponds to the slope field of the differential equation \[\frac{dy}{dx}=\frac{Q}{P}.\]
\vfill
Thus, a solution to this ODE describes the path taken by the particle in this fluid flow.
\end{observation}

\begin{activity}{10}
Consider the simpler ODE \[\frac{dy}{dx} = \frac{-2xy^2-1}{2x^2}.\]

This can be rewritten as \[ (2xy^2+1) + 2x^2 \frac{dy}{dx} = 0 .\]
\vfill
Now, consider \(\phi(x,y)=x^2y^2+y \).  
\begin{subactivity}
Compute \(\nabla \phi \).
\end{subactivity}
\begin{subactivity}
Differentiate the equation \(\phi(x,y)=c\) with respect to \(x\).
\end{subactivity}
\begin{subactivity}
Solve the ODE \( (2xy^2+1) + 2x^2 \frac{dy}{dx} = 0 \).
\end{subactivity}
\end{activity}

\begin{definition}
If \(\langle M,N\rangle\) is a conservative vector field, the ODE
\[M + N \frac{dy}{dx} = 0 \]
is called \term{exact}.  This ODE can also be written
\[\frac{dy}{dx} = \frac{ -M}{N} .\]
If \(\phi(x,y)\) is a potential function of \(\langle M,N\rangle\), the general solution to the
ODE is \(\phi(x,y)=c.\)
\end{definition}

\begin{activity}{10}
Determine which of the following ODEs are exact.
\begin{enumerate}[(a)]
\item \(2xy+(x^2-2y)\frac{dy}{dx}=0\) 
\item \(\frac{dy}{dx} = \frac{2xy}{x^2+2y} \)
\item \(\frac{dy}{dx} = -\frac{2xy}{x^2+2y} \)
\end{enumerate}
\end{activity}

\begin{activity}{10}
Solve the exact ODE \(2xy+(x^2-2y) \frac{dy}{dx}=0\).
\end{activity}

\begin{activity}{10}
Determine which of the following ODEs are exact.
\begin{enumerate}[(a)]
\item \(\frac{dy}{dx}= \frac{x}{x^2+y^2-y}\)
\item \( -\frac{x}{x^2+y^2} +(1-\frac{y}{x^2+y^2}\frac{dy}{dx} = 0\)
\end{enumerate}
\end{activity}

\begin{activity}{10}
Solve the exact ODE 
\[ -\frac{x}{x^2+y^2} +(1-\frac{y}{x^2+y^2}\frac{dy}{dx} = 0.\]
These solutions describe the trajectories taken by particles in the fluid flow in the first slide.
\end{activity}



\end{applicationActivities}
